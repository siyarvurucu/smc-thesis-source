\chapter{Conclusion and Future Work}
In this work, our goal was to improve the onset detection for online guitar courses, where the recordings are amateur (noisy, reverberated etc.). We studied the common noises in amateur recordings and developed an algorithm to perform better in such recordings. We compared our algorithm, Harmonic Onset Detector, on three separate datasets (GuitarSet, MusicCritic, Guitar Noises) against the state-of-the-art algorithm (CNNOnsetDetector) and the algorithm of MusicCritic (MC-OnsetDetector). Algorithms were evaluated with different offset and threshold values. Overall, Harmonic Onset Detector was found to be robust against noises and better at detecting guitar notes in beginner exercises and realistic recordings. We improved the automatic rhythm assessment predictions of MusicCritic dataset and expressed the importance of onset detection in automatic assessment. During the course of the work, an annotation and analysis tool was developed and a dataset of guitar noises was created. Both are available online (see appendix).

Although the new algorithm is developed with a focus on guitars, it can be adjusted to other pitched instruments too. In an online course setting, some parameters of the algorithm can be adjusted automatically according to the exercise. These adjustments would improve performance further. For example, information of tempo and the shortest note duration can be used to adjust the total length threshold of the segmentation step. Some parameters (e.g. smoothing window size) have different optimal values for chords and single notes. In our evaluations, we used the same value for both chord and solo recordings, which was non-optimal for both. So the type of exercise could also be used. Downsides of the algorithm are (1) It is computationally more expensive than other algorithms. (2) A playing technique where the intended sound is percussive and non-pitched (dead note) is neglected.

In many studies (in the field of MIR), development and validation parts of datasets come from the same distribution, so the algorithms are often biased on the dataset they are evaluated. In our study, we also showed that performances of onset detection algorithms highly depend on the application and context. In literature, the onset detection task is sometimes considered solved, at least for percussive (pitched or non-pitched) instruments. Our results show that the task is far from solved. The algorithms only work on trivial recordings and fail on real life scenarios. The main reason is that the description of musical onsets changes with the context, and a human can adapt to it easily. For example, in a noisy environment, knowing which instruments are played helps a human to detect the musical notes. A human-level universal onset detection system would require a source separation and identification algorithms. Such algorithms would require large datasets for both evaluation or training. 

Currently, lack of available datasets (especially on music performance analysis) is a major obstacle in the development of both onset detection and automatic assessment algorithms. One solution is the data augmentation techniques, as it is used on other tasks such as sound classification \cite{salamon2017deep}. Noises that can occur in non-ideal recording environments can be added to existing onset detection datasets. Mathematical models of instrument specific noises can be used to generate artificial noises for data augmentation. For guitars, there are models for slide noises \cite{pakarinen2007analysis} but not for buzz noises. Methods used for sitar sound synthesis \cite{vyasarayani2009} can be adapted to guitars to generate buzz noises. 

The standard evaluation method of onset detection was not effective in describing the behaviours of algorithms, which was discussed in the previous chapter. There is a need for new evaluation methods or metrics for better explanation and comparison of the onset detection algorithms.

We discussed PATs (perceived attack times) of strummed chords in related work and automatic rhythm assessment sections. We used the subjectivity argument to use the onset to onset differences instead of onset to metronome differences in rhythm assessment. For more accurate rhythm assessment and feedback, we still need to predict PATs of strummed chords. More studies with listening experiments are needed in order to develop a PAT prediction model. 

\newpage
