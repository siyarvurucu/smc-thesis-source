%\addcontentsline{toc}{chapter}{Abstract}

\begin{abstract}
\pagenumbering{gobble}
 
The aim of automatic musical performance analysis systems is to determine the students' grades or provide feedback to the students. Onset detection is the initial step of such systems. In literature on musical onset detection, complications of amateur recordings are not addressed. Due to the skill level of the player and non-ideal recording environments, there are plenty of noises in amateur recordings. Guitars are noisy instruments, especially on a beginner's hands. Existing onset detection algorithms do not perform well on amateur guitar recordings. This deteriorates the performance of automatic analysis systems. Our aim is to overcome this problem by developing an onset detection algorithm.

We study the common noises in guitar recordings and develop a new onset detection algorithm based on our observations. The developed algorithm and other selected algorithms are evaluated and compared on two datasets; GuitarSet (professional recordings) and MusicCritic dataset (amateur recordings). The effect of onset detection algorithms on automatic assessment is examined by using onset predictions to predict rhythm grades of recordings in MusicCritic dataset. 

Results show that our algorithm performs better than other algorithms including the state-of-the-art algorithm. This indicates the importance of including amateur recordings in the development and evaluation of onset detection studies, which is almost always neglected.

We discuss the standard evaluation method of onset detection and possible ways to improve it. We point out some future directions to improve onset detection for automatic performance assessment. Finally, a sound analysis tool and a dataset of guitar noises are made available online.

\bigskip


\newpage
\end{abstract}
