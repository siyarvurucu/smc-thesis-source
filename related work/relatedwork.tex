\chapter{Related Work}

Studies on automatic musical performance assessment usually adopt the existing algorithms for tasks such as chord detection, automatic transcription or onset detection. A recent overview of performance assessment studies is provided by Lerch et al. \cite{lerch2019mpa}. More general reviews of MIR and music education can be found in \cite{dittmar2012} and \cite{percival2007effective}. 

There are a few studies that develop a new onset detection algorithm, and only one for guitars \cite{eremenko2020performance}. There are no public work focusing on amateur (noisy) recordings or their implications in the context of MOOCs. In this chapter, we review onset detection algorithms in general, focusing on the ones used for guitars or performance assessment systems. 

\section{Onset Detection}

Onset can be defined as the first detectable part of a note event in the recording if the note were isolated \cite{leveau2004}. The task can be separated as offline and online (real-time) onset detection. Some applications provide real-time feedback to the player (e.g. karaoke, Rocksmith\footnote{www.rocksmith.com}, Yousician\footnote{https://yousician.com}) and require online detection. B{\"o}ck et al. \cite{bock2012evaluating} provides an overview for online onset detection. Music performance analysis systems do not necessarily require onset detection in real-time. In MusicCritic, analysis is done after performances are recorded. Therefore we focus on offline onset detection in the rest of the work.

Most existing algorithms can be grouped under signal processing, machine learning or probabilistic methods. There are several reviews \cite{hainsworth2003onset} \cite{collins2005comparison} \cite{bello2005tutorial} \cite{dixon2006onset} available mostly covering signal processing methods. Hidden Markov Models (HMM) are commonly used in several probabilistic methods \cite{abdallah2003unsupervised}, \cite{raphael2010music}.

Signal processing methods rely on spectral energy, phase, pitch, or a combination of those. A musical onset most likely increases the energy of the signal, which simply explains the motivation behind common usage of energy. However in complex situations such as a quiet note is played while another note is decaying, the total energy might not increase. This issue is addressed by discarding the frequencies that are losing energy in spectral flux \cite{spectralflux} (eq. \ref{SF}). Spectral flux is widely used within many other algorithms \cite{holzapfel2009three}, \cite{bock2013maximum}. Wu and Lerch \cite{wu2018learned} combined spectral flux with an adaptive peak picking method for their experiments in assessment of percussive instruments.\\
Spectral energy is often sufficient for detecting the onsets of percussive instruments but not for instruments with slow attacks, such as wind, bowed and voice. As first introduced \cite{bello2003phase}, phase information is found to be useful for non-percussive instruments. The energy of a note with a slow attack may increase steadily for a long duration, which makes it an imprecise indicator of the onset location. Whereas phases of frequencies change abruptly only in the beginning of the attack. However, abrupt changes in phase may arise due to the unreliability of phase processing \cite{holzapfel2009three} or inaudible noises. A common approach is to combine phase information with other onset detection functions. Bello et al. \cite{bello2004use} used both energy and phase information and reported overall improvement over the use of energy or phase only.

Pitch information is especially powerful on monophonic instruments with slow attacks and seldom unwanted noises, such as wind instruments. The absence of noises allows clear detection of pitch over contours. Two recent automatic assessment studies by Vidwans et al. \cite{vidwansobj} and Wu et al. \cite{wu2016towards} take the boundaries of the pitch contours as onsets for wind instruments. For more complex scenarios, pitch information can be combined with energy \cite{tan2010audio} \cite{zhou2007music} or both phase and energy \cite{brossier2004fast} \cite{holzapfel2009three}.

Vibrato and tremolo techniques create fluctuations on pitch and energy of a note. Those fluctuations cause multiple false detections. Vibrato suppression methods are developed on energy \cite{bock2013maximum} and pitch based \cite{collins2005using} detection algorithms to address this issue. 

Özaslan and Arcos \cite{ozaslan2010legato} focused on the identification of playing techniques legato and glissando on classical guitar. Plucked onsets are detected plucked notes with HFC and YIN pitch detection algorithm is used to detect the technique. Laurson et al. \cite{laurson2010simulating} worked on the simulation of the rasgueado technique on the classical guitar, where the notes are very close to each other due to fast strumming. The onsets from the real recording are detected by selecting the peaks of the smoothed total energy of frequencies between 11kHz and 20kHz. 

Mounir et al. \cite{mounir2016guitar} proposed an algorithm for guitar onset detection, which is called NINOS$^2$. After taking the STFT of the audio, the algorithm measures the sparsity of the spectral energy after discarding the frequencies with high energy. The motivation is that the low energy frequencies represent the guitar onsets better, as they usually arise from the interaction of finger (or plectrum) and strings and decay fast. The algorithm predicts the frames as onsets that have low sparsity.

Kehling \cite{kehling2014automatic} developed an automatic transcription system with an onset detection stage. Three existing onset detection algorithms (Spectral flux, Pitchogram Novelty \cite{abesser2017instrument} and Rectified Complex Domain \cite{dixon2006onset}) are applied and combined additively. Each algorithm exploits a different feature; energy, pitch and phase. The combination is found to be performing better than individual algorithms.

In MusicCritic \cite{eremenko2020performance}, Superflux \cite{bock2013maximum} algorithm is used. For the elimination of noises, detections are rejected if the energy difference is less than zero or averaged spectral centroid is more than a threshold. 

Neural network based algorithms, as in many other MIR tasks, perform best in onset detection. According to comparisons in MIREX \cite{mirex} \footnote{https://www.music-ir.org/mirex/wiki/MIREX\_HOME} in the last years, Convolutional Neural Network (CNN) \cite{lecun1998gradient} based algorithms achieve higher detection scores than previous algorithms. The current state-of-the-art onset detection algorithm (CNNOnsetDetector) is also a CNN developed by Schlüter and Böck \cite{schluter2014improved}. Their motivation behind the use of CNN on onset detection is that note onsets create edges in spectrograms and CNNs can learn to detect edges effectively.

\section{Perceived Attack Time}

Physical onset time (PhOT) is the actual acoustic beginning of an audio event, perceptual onset time (POT) is the moment listeners perceive the event and perceptual attack time (PAT) is defined as the perceived moment of rhythmic placement of the event \cite{wright2008shape}. Most onset detection studies aim to find PhOT or POT of musical events. Although physical onset is useful for analysis of the audio, PAT is more accurate for rhythmic performance assessment. Polfreman \cite{polfreman2013} evaluated nine different onset detection algorithms on five different onset types, concluding that the algorithms are not suitable to detect PAT of non-percussive sounds. 

On guitars, POTs of single notes are close to their PATs, since the instrument is plucked and percussive. This is not the case for strummed chords. A strummed chord is a single musical object that consists of multiple onsets close to each other. Hove et al. \cite{hove2007sensorimotor} showed that the PAT (they used the term perceptual center) of two close tones depends on the pitch of the tones, their order, and the amount of time between them. 
    
Frerie et al. \cite{freire2018strumming} studied the beat location of guitar strums perceived by the players. In their experiment, the same excerpts are played by different musicians on an acoustic guitar with hexaphonic pickups to record each string. Results showed that each player aligns chords to the metronome differently.
